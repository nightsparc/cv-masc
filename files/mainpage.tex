%---------------------------------------------------------------------------------------
%	PROFILE
%----------------------------------------------------------------------------------------
\cvsection{Profile + Status}
\vspace{4pt}

	\cvtext{
		\textbf{PhD candidate, M.Eng. Electrical Engineering and Information Technology, Software Engineer and Architect}

		\cvlist{
			\item Software engineer with a strong background in distributed systems and backend development with a passion
				OpenSource software.
			\item Focused on SW quality, efficiency, and maintainability, with a strong belief in KISS-principles.
			\item Strong advocate of agile mindsets und work environments.
			\item Customer-oriented and structured method of working.
		}
	}
%---------------------------------------------------------------------------------------
%	WORK EXPERIENCE
%----------------------------------------------------------------------------------------
\vspace{10pt}
\cvsection{Work experience}
\vspace{4pt}

	\cvevent{2019 -- now}
		{Lead Software Architect}
		{HAT.tec GmbH}
		{Lead development towards commercialisation and product development of AI-based human autonomy teaming technologies.}
		{
			\cvtext{
				\cvlist{
					\item System and software architecture design for human autonomy teaming and mission planning projects with
						  national and international partners.
					\cvlist{
						\item Design of th overall system and software architecture for the companies SW stack with the function specialists.
						\item Designed an AI-integrated SW architecture for environmental perception with future scaling capabilities (Cloud, K8s).
						\item Design of REST-APIs for platform-independent integration with partner software.
						\item Supervision and review of the software development of the individual modules in the SW stack.
					}
					\item Designed and developed a inter-process communication (IPC) shim, abstracting multiple IPC / middleware solutions (e.g. ROS2).
					\item Designed and developed several interface modules to interconnect with avionics and other
						third-party hardware modules.
					\item Adapted and launched (agile) software development and software lifecycle models to the company specifics
						and developed them towards ED-12C/DO-178C certifiable processes.
					\item Supervision of the AI-development team towards object detection, classification, and tracking.
					\item Ramping up the development team, technical suport in interviews etc.
					\item Part of the product strategy development team, outlining SW stack and architecture.
				}
			}
		}
		% \mbox{}
		% \vspace{2cm}
		\vfill\null
	%
	\cvevent{2012 -- 2019}
		{Research Associate \& Systems Engineer}
		{Institute of Flight Systems \newline Bundeswehr University Munich}
		{Research on cooperative system concepts for on-board environmental perception of teams of unmanned aerial vehicles (UAVs).}
		{
			\cvtext{Research interests: Multi-UAV Cooperation, Multi-Sensor-Fusion, Aerial Computer Vision, Manned-Unmanned Teaming (MUM-T)\\[8pt]
					Teaching: mission sensor classes + labs; theses supervision (BA/MA)}\\[8pt]
			\cvtext{
				\textbf{CASIMUS}\\[5pt]
				Researcher in the national research project \href{https://www.unibw.de/lft/projekte/casimus-lft}{CASIMUS} which
					investigated the deployment of multiple UAVs to support a manned two-seated transport helicopter with up-to-date recce
					data during the course of mission.
				\cvlist{
					\item Designed and -developed a hard- and software framework for the environmental perception on-board (multiple) UAVs.
					\item Integrated the system in a full-mission simulator to automatically reconnoiter potentially unsafe helicopter landing points.
					\item Planned, performend and evaluated a multi-week operator-in-the-loop experimental campaign.
					\item Flight-tested the perception system on-board multiple UAV-demonstrators in a down-sized setup
						to showcase and demonstrate the systems real-life cooperation and coordination mechanisms as a proof-of-concept.
					\item \iconhref{Youtube}{8}{IFS Manned unmanned Teaming for Future Helicopter Missions}{https://youtu.be/fSA-s22yFG8}{black}
				}
			}\\%
			\cvtext{
				\textbf{PROACTIVE}\\[5pt]
				Reseracher in the the EU-funded research project \href{https://cordis.europa.eu/project/id/285320}{PROACTIVE}
					which investigated the usage of multi-sensor networks as well as information fusion and reasoning mechanisms
					to detect and predict imminent terrorist attacks.
				\cvlist{
					\item Showcased the usage of Micro- and Mini-UAVs as deployable sensor platforms to be dynamically integrated in the multi-sensor network.
					\item Provided IPC and middleware mechanisms and supported international project partners during their sensor and system integration using that middleware.
				}
			}
		}
	\vfill\null
	%
	\cvevent{2010 -- 2012}
		{Research Engineer (Part-time)}
		{TH Aschaffenburg}
		{Development of a FPGA-based pedestrian detection system for smart intersections.}
		{
			\cvtext{
				Worked on low-latency real-time pedestrian segmentation and intention detection on Full-HD-images with computer vision and machine learning
					methods on FPGAs for intersection assistance to detect vulnerable traffic participants as part of national automotive research project
					\href{http://ko-fas.de/41-0-Ko-PER---Kooperative-Perzeption.html}{Ko-PER}.
				\cvlist{
					\item Development of a combined GPU/FPGA/PC framework for real-time computer vision.
					\item Adaption and implementation of computer vision and machine learning algorithms on FPGAs.
				}
			}
		}
	\vfill\null
	%
	\cvevent{2009 -- 2012}
	{Working Student / PLC-Programmer (Part-Time)}
	{LÖMI GmbH}
	{
		\cvlist{
			\item PLC-programming and automation of solvent recycling and PIM debinding plants.
			\item In-house and customer-side start-up with domestic and international assignments.
			}
	}
	{}
\vfill\null
	%
	\cvevent{2008}
		{Working Student / Intern}
		{Reis GmbH \& Co. KG Maschinenfabrik}
		{Working student and internship as part of the Bachelor studies.}
		{
			\cvtext{
				\cvlist{
					\item Control hardware and power electronics development for industrial robots.
					\item EMC measurements of industrial robot systems and design of appropriate mitigation strategies.
					\item Programming and start-up of industrial robot systems for PV production plants.
					\item Test and evaluation of new CAE/CAD software solutions.
				}
			}
		}
	\vfill\null
	%
	\cvevent{2005 -- 2008}
		{Working Student / PLC-Programmer (Part-Time)}
		{LÖMI GmbH}
		{PLC-programming and assembly of solvent recycling and PIM debinding plants with in-house startup.}
		{}
	\vfill\null
	%
	\cvevent{2002 -- 2005}
		{Apprantice Electrician (Energieelektroniker)}
		{Integtronik GmbH}
		{Building customized industrial computers with my own hands :-)}
		{}
	\vfill\null

%---------------------------------------------------------------------------------------
%	EDUCATION
%----------------------------------------------------------------------------------------
\cvsection{Education}
\vspace{4pt}

	\cvevent{2012 - 2019}
		{PhD in Aerospace Engineering}
		{{Institute of Flight Systems (Prof. Stütz)\newline Bundeswehr University Munich}}
		{Working Title: \textit{\glqq A Cooperative Multi-UAV Perception Management System for the Highly-Automated Reconnaissance of Helicopter Lading Zones\grqq}}
		{}
	\vfill\null
	%
	\cvevent{2010 - 2012}
		{Master of Engineering (M.Eng.), Grade 1.1}
		{TH Aschaffenburg}
		{Master's thesis: \textit{\glqq Memory Management Concepts for Feature Extraction and Classification of Vulnerable Traffic Participants\grqq}}
		{\cvtext{Focussed on real-time Computer Vision and Machine Learning}}
	\vfill\null
	%
	\cvevent{2006 - 2010}
		{Bachelor of Engineering (B.Eng.), Grade 1.6}
		{TH Aschaffenburg}
		{Bachelor's thesis: \textit{\glqq Implementation of a DDR-RAM Controller for Computer Vision Tasks on an FPGA\grqq}}
		{\cvtext{Electrical engineering and information technology with a focus on automation technologies and programmable hardware.}}
	\vfill\null
%---------------------------------------------------------------------------------------
%	PUBLICATIONS
%----------------------------------------------------------------------------------------
% \cvsection{Publications}

% \begin{itemize}[leftmargin=*]
% 	\item Author, G. \& Author, P. \&  Author G. (2020). "This is the title of the publication". In: \textit{Proceedings of the 28th Conference on Lipsum (LIPSUM)}, Lipsum, June 12-17, 2020.
% \end{itemize}
% hofixes to create fake-space to ensure the whole height is used
\mbox{}
\vfill
\mbox{}
\vfill
\mbox{}
\vfill
\mbox{}
\vfill
\mbox{}
\vfill

\hspace*{5.5cm}\includegraphics[width=4.5cm]{resources/unterschrift_klein}\\[-6pt]
Munich, \today     \hspace{1cm}   \hrulefill

\hspace*{30mm}\phantom{Munich, \today }\name